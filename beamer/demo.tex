\documentclass[8pt, utf8, a4paper, handout]{beamer}

\usepackage[english]{babel}
\usepackage[T1]{fontenc}

\usepackage{acmebeamer}
\usepackage{arev}

\usepackage{fancyvrb}

\usetikzlibrary[shapes.geometric]

\acmeplaintheme

\setbeamersize{text margin left=1em, text margin right=1em}
\setbeamerfont{normal text}{size=\footnotesize}
\setbeamerfont{itemize/enumerate subbody}{size=\footnotesize}%WTF
\setbeamerfont{subsection in toc}{size=\footnotesize}
\setbeamerfont{block title}{size=\normalsize, parent={structure,block body}}

\def\bold{\simplegroupedcommand{\bfseries}{}}
\def\slanted{\simplegroupedcommand{\slshape}{}}

\leftmarginii=.9em
\leftmarginiii=.8em

\setbeamertemplate{itemize item}
  {\unitsquare[fill, scale=3, yshift=.3]}
\setbeamertemplate{itemize subitem}
  {\unittriangle[fill, scale=3, yshift=.3]}
\setbeamertemplate{itemize subsubitem}
  {\unittriangle[draw, scale=2, yshift=.5]}

%<< [·Handy commands for documentation·] >>

% \def\docsetbeamer#1#2{\btype{setbeamer#1\{#2\}\{…\}}}
% \def\docusebeamer#1#2{\btype{usebeamer#1\{#2\}\{…\}}}

% \edef\longlorem{Lorem ipsum dolor sit amet, consectetur adipiscing elit.
%   Praesent tincidunt est nec odio ultricies fringilla tempor et lectus.
%   Duis accumsan varius nisl placerat volutpat. Quisque feugiat, sapien
%   sit amet ultricies cursus, tellus odio auctor purus, sed placerat quam
%   enim at urna. Nunc molestie magna eu turpis fringilla ac ultricies
%   ante convallis. Proin sed magna eros. Aliquam lobortis sapien eu augue
%   faucibus hendrerit. Class aptent taciti sociosqu ad litora torquent
%   per conubia nostra, per inceptos himenaeos. Nulla facilisi. Vivamus ut
%   purus non metus facilisis convallis. Pellentesque bibendum purus nec
%   quam molestie ultrices. Maecenas turpis odio, volutpat quis aliquet a,
%   ultrices aliquam neque. Aenean eleifend imperdiet felis a viverra. Sed
%   semper vulputate volutpat. Pellentesque sit amet orci tellus, ut
%   imperdiet metus. Sed et augue a augue suscipit facilisis. Proin
%   malesuada dolor non ligula dictum molestie.}

% \edef\mediumlorem{Lorem ipsum dolor sit amet, consectetur adipiscing
%   elit. Praesent tincidunt est nec odio ultricies fringilla tempor et
%   lectus. Duis accumsan varius nisl placerat volutpat. Quisque feugiat,
%   sapien sit amet ultricies cursus, tellus odio auctor purus, sed
%   placerat quam enim at urna. Nunc molestie magna eu turpis fringilla ac
%   ultricies ante convallis. Proin sed magna eros. Aliquam lobortis
%   sapien eu augue faucibus hendrerit.}

\edef\smalllorem{Lorem ipsum dolor sit amet, consectetur adipiscing
  elit. Duis accumsan varius nisl placerat volutpat.}

% \edef\tinylorem{Lorem ipsum dolor sit amet.}

% \let\lorem\mediumlorem

\def\vdelimiton{\def\vdelimit{\vrule}}
\def\vdelimitoff{\def\vdelimit{\vrule width0pt}}

\vdelimitoff

\def\Describe{\dosingleempty\doDescribe}
\long\def\doDescribe[#1]#2#3{%
  \pgfkeys{/describe/.cd,
    leftwidth/.initial=.2\textwidth,
    rightwidth/.initial=.8\textwidth,
    leftstyle/.initial=\type,
    rightstyle/.initial=, #1}%
  \blank[line]
  \begin{columns}[onlytextwidth, t]
    \vdelimit
    \begin{column}{\pgfkeysvalueof{/describe/leftwidth}}
      \bgroup\pgfkeysvalueof{/describe/leftstyle}#2\egroup
    \end{column}\vdelimit
    \begin{column}{\pgfkeysvalueof{/describe/rightwidth}}
      \parskip=.5\baselineskip plus.5\baselineskip minus.5\baselineskip
      \bgroup\pgfkeysvalueof{/describe/rightstyle}#3\egroup
    \end{column}\vdelimit
  \end{columns}}

\newcommand\samplecode[3][]{%
\blank[2*line]
\begin{columns}[onlytextwidth]
  \begin{column}{.475\hsize}
    #2
  \end{column}
  \begin{column}{.475\hsize}
    \usage{#3}
  \end{column}
\end{columns}}

\newcommand\warningsign[1][]{%
  \begin{tikzpicture}[draw=darkred, fill=white, #1]
    \node [scale=1.3, shape border rotate=90,
           shape=isosceles triangle,
           isosceles triangle apex angle=60,
           draw, fill, line width=3pt,
           rounded corners, font=\Large\bfseries] at (6,0) {!};
  \end{tikzpicture}}

\newcommand\notice[2][]{%
\begin{noticeblock}[#1]#2\end{noticeblock}}

\newenvironment{noticeblock}[1][]{%
  \hbox to \hsize
  \bgroup\hss
  \begin{framedtext}
    [width=.8\hsize, minheight=35pt, background=on,
     tikzoptions={draw=darkred, fill=darkred!10, very thick,
       rounded corners=2pt}, #1]
    \begin{columns}[totalwidth=\hsize]
      \begin{column}{.1\hsize}
        \warningsign[baseline=-12pt]%
      \end{column}
      \begin{column}{.85\hsize}\bgroup}{%
      \egroup\end{column}
    \end{columns}
  \end{framedtext}%
  \hss\egroup}

\def\type{%
  \simplegroupedcommand
  {\scriptsize\ttfamily\let\\\textbackslash\parskip=0pt}
  {}}
\def\usage{%
  \simplegroupedcommand
  {\usebeamercolor[fg]{emphasis}\scriptsize\ttfamily\let\\\textbackslash}
  {}}

\def\btype{\simplegroupedcommand{\type\textbackslash}{}}
\newcommand\hbtype[2][\bfseries]{\btype{#1#2}}

\def\benvtype#1{\btype{begin\{#1\}}}
\def\eenvtype#1{\btype{end\{#1\}}}
\def\beenvtype#1#2{\btype{begin\{#1\}}\type{#2}\btype{end\{#1\}}}

\def\ctype#1#2{\btype{#1\{#2\}}}
\newcommand\hctype[3][\bfseries]{\ctype{{#1#2}}{#3}}

\def\dtype#1#2#3{\btype{#1\ctype{#2}{#3}}}
\def\deftype{\dtype{def}}

%>>


\title{Documentation of the {\tt acmebeamer} package.}
\subtitle{A presentation like/showcase document.}

\author{Cédric Mauclair}
\institute{Onera/DTIM}

\date[2011-02-07]{February 7, 2011}




\begin{document}

\vdelimitoff

\setuptitlepage[lineoffset=.75\paperheight]

\maketitle[logo=institute]

\frametoc{Table of contents}

\def\blankline{\blank[line]}


\section{Package options}
%<< (folded) >>

\begin{frame}[fragile,fragile]

  \Describe{portrait}%
  {Swaps the height and width of the document.}

\end{frame}

%>>



\section{Beamer features}

\subsection{Title page}
%<< (folded) >>

\begin{frame}

  \begin{sitemize}
  \item {\bold Usual and basic way}

    \Describe{\\maketitle}%
    {Modified to typeset the macro \btype{inserttitlepage} inside
      a \type{frame} environment.}

    \Describe{\btype{titlegraphic}}%
    {Sets the value of \btype{inserttitlegraphic}.}

    \Describe{\btype{setbeamertemplate}}%
    {Define the template to use for the title page. Same as
      \btype{titlegraphic} in this context.}

  \blank[2*big]
  \item {\bold Going a bit further}
    \begin{sitemize}
    \item \btype{maketitle} actually takes {\bold two optional
        arguments}. The first one is passed to \btype{setuptitlepage}
      (see below), the second one is passed to the frame options. If you
      want to give the second argument without the first one, you must
      give an explictly empty first argument.

      Use as%
      \hfill\usage{\\maketitle%
        [{\slanted <options to \\setuptitlepage[…]>}]%
        [{\slanted <options to \\begin\{frame\}[…]>}]}.

    \item The package provides a way to setup the title page with the
      macro \btype{setuptitlepage[…]}, according to a few (for now)
      options of the default template. There are three options as of
      now, \type{lineoffset}, \type{logo} and \type{customlogo}.
      \begin{sitemize}
      \item \type{lineoffset}: sets the height of the line in the
        default template.
      \item \type{logo}: two values only (for now), \type{none} and
        \type{institute}. One way guess what they do.
      \item \type{customlogo}: sets its value to be the logo. The last
        one wins.
      \end{sitemize}
    \end{sitemize}
  \end{sitemize}

\end{frame}

%>>

\subsection{Table of contents}
%<< (folded) >>

\begin{frame}

  \Describe{\\frametoc}%
  {Takes one mandatory argument (in braces) {\bold and then}, an
    optional argument (in brackets). The mandatory argument is the title
    of the frame. The optional argument is passed
    to \btype{tableofcontents}.

    Use as%
    \hfill\usage{\\frametoc%
      \{{\slanted <frame title>}\}%
       [{\slanted <options to \\tableofcontents[…]>}]}.}

\end{frame}

%>>

\subsection{Page numbers}
%<< (folded) >>

\begin{frame}

  \Describe{\\setuppagenumbers}%
  {Takes one mandatory argument in square brackets, a key value list
    consisting of two possible keys: \type{style} and \type{custom}. The
    latter sets the page numbers command to typeset its argument. There
    are three values possible (for now) for the \type{style} key:
    \type{none}, \type{boxed} and \type{plain}. The last two typeset the
    pages as \type{\slanted <page number>/<total number of pages>}.

    Use as%
    \hfill\usage{\\setuppagenumbers[…]}.}

\end{frame}

%>>

\subsection{Head-~and footline}
%<< (folded) >>

\begingroup

\setupheadline[left=section, center/subsection=bullets]
\setupfootline[left=subtitle, center=author, right=date]

\begin{frame}

  The \type{acmebeamer} package provides a flexible way to customize the
  head-~and footline through two commands \btype{setupheadline} and
  \btype{setupfootline} described below.

  \blank[line]
  \Describe{\\setupheadline\par\\setupfootline}%
  {Takes one mandatory argument in square brackets, a key value list.
    There are four key available: \type{left}, \type{center},
    \type{right} and \type{frame}. The first three customize the
    respective places and \type{frame} is used to customize the width
    and the margins.

    You can set those with %
    \btype{setupheadline[frame=\{width=.8\\hsize, margin=3pt,
      topmargin=5pt\}]} for example. Available are \type{width},
    \type{margin}, \type{leftmargin}, \type{rightmargin},
    \type{topmargin} and \type{bottommargin}.

    The other options have the same behaviour and options. The values
    start being \type{empty}.
    \begin{sitemize}
    \item Standard options from \type{beamer}: \type{title},
      \type{subtitle}, \type{author}, \type{date}, \type{section} and
      \type{section}.
    \item New options, the basic way: \type{page numbers} and
      \type{empty}.
    \item New options, the tricky way: don't use \type{right=...}, but
      \type{right/subsection=...} with the following options available:
      \type{text} (same as doing \type{right=subsection}),
      \type{bullets} and \type{squares}. At some point, the package may
      offer a direct solution.
    \end{sitemize}}

  \blank[2*line]
  \begin{columns}[onlytext]
    \begin{column}{.5\hsize}
      The present document uses\\
      \usage{\\setupheadline[left=title, right=page numbers]}\\
      \usage{\\setupfootline[]}.
    \end{column}
    \begin{column}{.5\hsize}
      This particular slide uses\\
      \usage{\\setupheadline[left=section, center/subsection=bullets]}\\
      \usage{\\setupfootline[left=subtitle, center=author, right=date]}.
    \end{column}
  \end{columns}

  One can notice that the short forms are used for the standard options
  and that previous setups are not erased (see the page numbers on the
  right). If all \type{right}, \type{center} and \type{left} are empty,
  the head-~or footline doesn't occupy any space.
\end{frame}

\endgroup

%>>

\subsection{Frametitle}
%<< (folded) >>

\begin{frame}

  \framesubtitle{Great feature added!}

  \Describe{{\usebeamer{normal text}template} frametitle}%
  {Has been modified so that if the frame title is empty, then the
    template is not used. Besides, the subtitle always occupies space if
    the template is used (may change in the future).}

  \Describe{\\begin\{frame\}\par…\par\\end\{frame\}}%
  {Has been modified to be able to omit the title if one wants to use
    the same as the current section or subsection. Here are the rules:
    \begin{sitemize}
    \item No title provided:
      \begin{sitemize}
      \item uses the subsection (\btype{subsecname}) if not empty;
      \item or uses the section (\btype{secname}) if not empty;
      \item or uses the template with an empty title (if a subtitle is
        provided, it will be typeset).
      \end{sitemize}
    \item A title has been provided: normal behaviour. As a remainder,
      one can specify a title and a subtitle according to the following:
      \begin{sitemize}
      \item as the first and second {\bold braced group} if nothing
        happens before the groups except for spaces and comments
        \hfill\usage{\\begin\{frame\}
          \{{\slanted <title>}\}
          \{{\slanted <subtitle>}\}};
      \item as arguments to the commands \btype{frametitle} and
        \btype{framesubtitle} inside the \type{frame} environment.
      \end{sitemize}
      How can we add a subtitle to a frame that has the same title as
      the (sub)section? This slide uses
      \hfill\usage{\\begin\{frame\} \\framesubtitle\{…\}}.

      There is a small catch however: you must use \btype{maketitle} or
      you can use \btype{acmeframetitlehack} before the frames you want
      to use this feature on. All the following frames will use the
      feature.
    \end{sitemize}}


\end{frame}

%>>



\section{Extra features}

\subsection{Better {\tt itemize} environment}
%<< (folded) >>

\begin{frame}[fragile]

  \Describe{\\begin\{sitemize\}\par…\par\\end\{sitemize\}}%
  {Sometimes, we would like to list a bunch of things, but we do not
    want to start at level~1. This is the {\slanted raison d'être} of
    this environment. You can specify the level in an optional
    argument: allowed values are~\type{1}, \type{2}~and~\type{3}. The
    values are {\bold relative to the current nesting level}. The
    default level is~\type{1} for compatibility reasons.}

  \blank[3*line]
  \begin{columns}[t, onlytext]
    \begin{column}{.3\hsize}
      Example\blank[1pt]
      \begin{enumerate}[<+->] % remove 'handout' to see
      \item Text goes here.
        \begin{sitemize}
        \item one-one (third level)
        \item one-two (third level)
        \end{sitemize}
      \item Text goes here.
        \begin{sitemize}
        \item two-one (second level)
        \item two-two (second level)
          \begin{enumerate}[a]
          \item a
          \item b
          \end{enumerate}
        \item two-three (second level)
        \end{sitemize}
      \end{enumerate}
    \end{column}
    \begin{column}{.3\hsize}
      Typeset with
      \bgroup\usage
\begin{verbatim}
\begin{enumerate}[<+->] % remove 'handout' to see
\item Text goes here.
  \begin{sitemize}
  \item one-one (third level)
  \item one-two (third level)
  \end{sitemize}
\item Text goes here.
  \begin{sitemize}
  \item two-one (second level)
  \item two-two (second level)
    \begin{enumerate}[a]
    \item a
    \item b
    \end{enumerate}
  \item two-three (second level)
  \end{sitemize}
\end{enumerate}
      \end{verbatim}\egroup
    \end{column}\hfill
  \end{columns}

\end{frame}

%>>

\subsection{Blocks on steroids}
%<< (folded) >>

\begin{frame}
  {\subsecname \hfill \slide 1 of 4}

  \begin{noticeblock}
    Blocks are completely rewritten from scratch. They are still
    accessible with the same commands, but the arguments are somewhat
    different and are given differently too.
  \end{noticeblock}

  \begin{sitemize}
  \item {\bold \#1 the arguments}

    The beamer blocks take only one optional parameter {\bold in braces}
    which is the title of the block, if any. Now, there is two optional
    arguments, one in braces, the title, {\bold and then} one in square
    brackets. The latter are options destined to the new system of
    blocks.

    \hfill\usage{%
      \\begin\{{\slanted blockname}\} \{{\slanted <title>}\}
              [{\slanted <options>}]…
      \\end\{{\slanted blockname}\}}.



  \item {\bold \#2 basic options}\blank[-5pt]
    \Describe{width}{Sets the width of the block.}

    \Describe{color}{Sets the color of the block.}

    \Describe{align}%
    {Sets the alignment of the block, not the text inside the block (see
      below).}

    \Describe{bodyheight}{Sets the height of the blockbody.}

    \Describe{title/bodycolor}%
    {Sets the color of the title/body, this can be a beamer color or
      just a color (like ``yellow'').}

    \Describe{title/bodystyle}%
    {Sets the style of the title/body (like ``\type{\\bfseries}'' or
      ``\type{\\slshape\\large\\color\{green\}}'').}

    \Describe{title/bodyalign}%
    {Sets the alignment of the text inside. Possible values are:
      \type{default}, \type{flushleft}, \type{flushright} and
      \type{center} horizontaly and \type{top}, \type{bottom} and
      \type{middle} vertically. You can give several (typically two) as
      a list in braces: \type{bodyalign=\{center,top\}}. Better avoid
      spaces.}



  \item {\bold \#3 advanced options}\blank[-5pt]
    \Describe{title-/bodycustomframe\par title-/bodybackground}%
    {Sets the frame Ti{\slanted k}Z path. You can disable the frame with
      \type{false} or anything that will be ignored in
      a \type{tikzpicture} environment. So \type{true} alos works, but
      is confusing. The same goes for the background. (Defaults on
      third next slide.)}

    \Describe{title-/bodyoptions}%
    {Options passed to the \type{framedtext} environment. (See next
      section.)}
  \end{sitemize}

\end{frame}


\begin{frame}[fragile]
  {\subsecname \hfill \slide 2 of 4}
  {Examples}

  \samplecode
  {\begin{block}
     {With a yellow title}
     [titlecolor=yellow]
     \smalllorem
   \end{block}}
  {\\begin\{block\} \{With a yellow title\} \par
   ~~[titlecolor=yellow]                    \par
   ~~\\smalllorem                           \par
   \\end\{block\}}

  \samplecode
  {\begin{alertblock}
     [bodyalign=center, bodystyle=\bold]
     Without title, bold centered body text.
   \end{alertblock}}
  {\\begin\{alertblock\}                      \par
   ~~[bodyalign=center, bodystyle=\\bfseries] \par
   ~~Without title, bold centered body text.  \par
   \\end\{alertblock\}}

  \samplecode
  {\begin{exampleblock}
     {Left aligned title}
     [titlealign=flushleft, bodyheight=20pt,
      bodyalign={bottom,flushright}]
     Fixed body height to 20pt, bottom right aligned.
   \end{exampleblock}}
  {\\begin\{exampleblock\} \{Left aligned title\}         \par
   ~~[titlealign=flushleft, bodyheight=20pt,              \par
   ~~~bodyalign=\{bottom,flushright\}]                    \par
   ~~Fixed body height to 20pt, bottom and right aligned. \par
   \\end\{exampleblock\}}

  \samplecode
  {\begin{emphasisblock}
     {A new one, \type{\small emphasisblock}}
     \begin{sitemize}
     \item one
     \item two
     \item three
     \end{sitemize}
   \end{emphasisblock}}
  {\\begin\{emphasisblock\} \{A new one, emphasisblock\} \par
   ~~\\begin\{sitemize\}                                 \par
   ~~\\item one                                          \par
   ~~\\item two                                          \par
   ~~\\item three                                        \par
   ~~\\end\{sitemize\}                                   \par
   \\end\{emphasisblock\}}                               \par

\end{frame}


\begin{frame}
  {\subsecname \hfill \slide 3 of 4}
  {The case of lists and enumerations}

  The use of lists in blocks is somewhat screwed up if there is text
  around. The package offers a dirty solution in the form of a macro to
  use between the text and the \type{\\(s)itemize/enumerate}
  environments: \btype{fixvspace}.

  \begin{block}[width=.45\hsize]
    Before text: \btype{fixvspace} macro between text and
    environment.\fixvspace
    \begin{sitemize}
    \item one
    \item two
    \end{sitemize}
    After text: no \btype{fixvspace} macro between text and environment.
  \end{block}
  \begin{block}[width=.45\hsize]
    \begin{sitemize}
    \item one
    \item two
    \end{sitemize}
  \end{block}

\end{frame}


\begin{frame}
  {\subsecname \hfill \slide 4 of 4}
  {Default setup}

  \Describe{\\defineblock[…][…]\par\\defineblocks[…][…]}%
  {Defines a block or several at once, either from an existing block, or
    a setup.}

  \Describe{\\setupblock[…][…]\par
    \\setupblocks[…][…]\par\\setupblocks[…]}%
  {Setups a block, several blocks at once or setup the default for all
    blocks. Unless specified otherwise, blocks inherit their parameters
    from \type{blocks}. If defined from another block, they inherit from
    it. All parameter set explicitely is specific to the block.}

  \blank[line]
  \begin{columns}[t]
    \begin{column}{.35\hsize}
      \bgroup\usage\parskip=0pt
      ~~\\defineblock                                                \par
      ~~~~[default] [color=structure]                                \par
      ~~\\defineblock                                                \par
      ~~~~[alert] [color=alerted text]                               \par
      ~~\\defineblock                                                \par
      ~~~~[example] [color=example text]                             \par
      ~~\\defineblock                                                \par
      ~~~~[emphasis] [color=emphasis]                                \par
      \blank[line]
      ~~\% color of the bullets                                      \par
      ~~\\setbeamercolor \{defaultblockitems\}                       \par
      ~~~~\{parent=structure\}                                       \par
      ~~\\setbeamercolor \{alertblockitems\}                         \par
      ~~~~\{parent=alerted text\}                                    \par
      ~~\\setbeamercolor \{exampleblockitems\}                       \par
      ~~~~\{parent=example text\}                                    \par
      ~~\\setbeamercolor \{emphasisblockitems\}                      \par
      ~~~~\{parent=emphasis\}                                        \par
      \blank[line]
      ~~\% map defaultblock to LaTeX block                           \par
      ~~\\let\\defaultblock\\block                                   \par
      ~~\\let\\defaultendblock\\endblock
      \blank[line]
      \% example                                                     \par
      ~~\\defineblock                                                \par
      ~~~~[newexpample] [example]                                    \par
      ~~\\defineblocks                                               \par
      ~~~~[one,two, three] [default]                                 \par
      ~~\\defineblocks                                               \par
      ~~~~[four,five,six] [width=.5\\hsize]                          \par
      \egroup
    \end{column}
    \begin{column}{.62\hsize}
      \bgroup\usage\parskip=0pt
      \\setupblocks[\% no spaces after [ or before commas                            \par
      ~~width=\\hsize,                                                               \par
      ~~align=center,                                                                \par
      ~~customframe=                                                                 \par
      ~~  \{\\path [draw] (0,0) rectangle ++(\\framedboxwd,-\\framedboxht);\},       \par
      ~~custombackground=                                                            \par
      ~~  \{\\path [fill] (0,0) rectangle ++(\\framedboxwd,-\\framedboxht);\},       \par
      ~~frameoptions=                                                                \par
      ~~  \{offset=0pt,                                                              \par
      ~~   tikzoptions=\{draw=fg, fill=bg, thick, rounded corners=2pt\}\},           \par
      ~~titlecustomframe=none,                                                       \par
      ~~titlecustombackground=                                                       \par
      ~~  \{\\path [fill=fg]                                                         \par
      ~~  \{[sharp corners] (0,0) --- ++(0,-\\framedboxht) --- ++(\\framedboxwd,0)\} \par
      ~~  \{[rounded corners=2pt] --- ++(0,\\framedboxht)  --- cycle\};\},           \par
      ~~titleoptions=\{offset=4pt, bottomoffset=2pt, left=\\strut\},                 \par
      ~~titlecolor=white,                                                            \par
      ~~titlestyle=\\usebeamerfont\{block title\},                                   \par
      ~~titlealign=\{center\},                                                       \par
      ~~bodycustomframe=none,                                                        \par
      ~~bodycustombackground=false,                                                  \par
      ~~bodyoptions=\{offset=4pt\},                                                  \par
      ~~bodycolor=normal text,                                                       \par
      ~~bodystyle=\\usebeamerfont\{block body\},                                     \par
      ~~bodyheight=-\\maxdimen,                                                      \par
      ~~bodyalign=\{default, middle\}]
      \blank[line]
      \% example                                                                     \par
      \\setupblocks[default,alert][align=left]
      \egroup
    \end{column}
  \end{columns}

\end{frame}

%>>



\section{One more thing~…}

%<< (folded) >>

\begin{frame}
  {\secname \hfill \slide 1 of 2}
  {The {\tt framed} command and the {\tt framedtext} environment}

  The \type{acmebeamer} package relies on a package named
  \type{acmetoolbox} which provides (among other things) facilities to
  frame text similarly to packages like \type{bclogo} or
  \type{fancybox}. These facilities come as two commands for short
  snippets and an environment. They rely on Ti{\slanted k}Z to do the
  framing.

  \Describe{\\framed}%
  {Frames a text with a \btype{hbox}. It takes one optional argument in
    square brackets and one mandatory argument which is the text to
    frame. Use as \hfill\usage{\\framed[{\slanted <options>}]\{…\}}.}

  \Describe{\\inframe}%
  {Ditto as \btype{framed}, except the baseline of the text inside is
    the same as the text outside.}

  \Describe{\\beging\{framedtext\}\par…\par\\end\{framedtext\}}%
  {Frames a text with a \btype{vbox}. It takes one optional argument in
    square brackets.}

  You actually already saw some framed text: %
  \inframe[frame=off, background=on, tikzoptions={fill=green!35}]
  {the notice on slide~13.} This was an example of inframed text.

  \vfill\vfill{\usebeamercolor[fg]{structure}\hrule}\vfill

  There is yet another (convenient) way to highlight text in paragraphs
  and formulas.

  \Describe{\\hl}%
  {Highlights text. This command is overlay-aware!%
    \hfill\usage{\\hl[{\slanted <tikz options>}]\{…\}}}

  \Describe{\\placehlmark}%
  {Place a named mark. The name is optional and is specified in square
    brackets. The mandatory argument is placed in a Ti{\slanted k}Z node
    named after the name supplied (\type{mark} by default).%
    \hfill\usage{\\placehlmark[{\slanted <name of mark>}]\{{\slanted
        <text to be marked>}\}}}

  \Describe{\\hlmarks}%
  {\placehlmark[mark1]{\hl{Used to highlight several marks.}} This
    command is overlay-aware too! An optional argument can be supplied
    in square brackets to indicate the shape to use to hightlight the
    marks, this is \placehlmark[mark2]{\hl{\type{ellipse} by default.}}
    (See the source to understand how it is used.)

    \hfill\usage{\\hlmarks[{\slanted <tikz options>}]\{(some) (marks)
      (to) (ellipse)\}}}

  \hlmarks{(mark1) (mark2)}

\end{frame}


\begin{frame}
  {\secname \hfill \slide 2 of 2}
  {The available options}

  \Describe{left, right,\par top, bottom}%
  {Put something at the left, right, top and bottom respectively arround
    the text (which will be boxed by that time).}

  \Describe{left/right/\par top/bottom/frame}%
  {Take values \type{on} and \type{off} to activate/deactivate the left,
    right, top, bottom or the whole frame. The code is clever enough to
    use the whole frame if all sides are \type{on}.}

  \Describe{left/right/\par top/bottom/margin}%
  {Sets the margins around the frame (in addition to the width).}

  \Describe{left/right/\par top/bottom/offset}%
  {Sets the offsets inside the frame (part of the width).}

  \Describe{width, minwidth, maxwidth}%
  {Sets the width, minimum or maximum width of the resulting frame.}

  \Describe{height, minheight, maxheight}%
  {Sets the height, minimum or maximum height of the resulting frame.}

  \Describe{align}%
  {Defines the alignment of the text in the frame. (See title/bodyalign
    in previous slides.)}

  \Describe{background}%
  {Takes values \type{on} and \type{off} to activate or deactivate the
    background.}

  \Describe{customframe}%
  {Custom Ti{\slanted k}Z path to be used to draw the frame.\par
    Default is
    \hfill\usage{\\path [draw] (0,0) rectangle
      ++(\\framedboxwd,-\\framedboxht);}}

  \Describe{custombackground}%
  {Custom Ti{\slanted k}Z path to be used to draw the background.\par
    Default is
    \hfill\usage{\\path [fill] (0,0) rectangle
      ++(\\framedboxwd,-\\framedboxht);}}

  \Describe{tikzoptions}%
  {Options passed to the \type{tikzpicture} environment used to draw the
    frame.}

\end{frame}

%>>


\end{document}
