\documentclass{article}

\usepackage[english, french]{babel}
\usepackage[utf8]{inputenc}
\usepackage[T1]{fontenc}

\usepackage{acmetoolbox}
% \usepackage{acmebeamer, arev}

% \acmeplaintheme

\makeatletter
\parindent=0pt

\def\mycolor{red}

% \getparameters[root=acme, familly=framed][halign=\centering]
\pgfkeys{/acme/framed/.cd, .set root for alias=/def, halign=center,
  margin=12pt, tikzoptions={draw, fill=\mycolor}}


\def\mycolor{blue}

\begin{document}

  Texte de démonstration: rien de spécial !\vskip10pt

  \vrule\boxcontent[wd=120pt, align=\raggedleft]%
  \hbox{Small text.}\vrule

  \vskip20pt
  \vrule\boxcontent[align=\raggedleft]%
  \vbox{Hello toto, have a good! Hello toto, have a good! Hello toto,
    have a good! Hello toto, have a good! Hello toto, have a good! Hello
    toto, have a good! Hello toto, have a good! Hello toto, have a good!
    Hello toto, have a good!\par Next paragraph!}\vrule

  \vskip20pt%
  \noboxcontent[wd=\hsize-30pt]%
  \vbox{\vskip10pt Hello toto, have a good! Hello toto, have a good!
    Hello toto, have a good! Hello toto, have a good! Hello toto, have
    a good! Hello toto, have a good! Hello toto, have a good! Hello
    toto, have a good! Hello toto, have a good!\par\vskip5pt Next
    paragraph!\vskip10pt}

  \surroundbox%
    [lft=\vrule\hskip15pt, rgt=\hskip15pt\vrule, top=\hrule, bot=\hrule]%
  \box\nextbox%
  \flushnextbox

  \vskip20pt
  \surroundbox[lft=\vrule, rgt=\vrule, top=\hrule, bot=\hrule]%
  \vbox{Hello toto, have a good! Hello toto, have a good! Hello toto,
    have a good! Hello toto, have a good! Hello toto, have a good! Hello
    toto, have a good! Hello toto, have a good! Hello toto, have a good!
    Hello toto, have a good!\par Next paragraph!}
  \flushnextbox


  \acmeframedhalign
  Un texte quelconque. %\acmeframedleftmargin\ \acmeframedtikzoptions

  \vskip20pt
  \vrule\framed[margin=10pt, leftmargin=0pt, ]{toto}\vrule

\end{document}
